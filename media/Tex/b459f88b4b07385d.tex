\documentclass[preview]{standalone}

\usepackage[english]{babel}
\usepackage{amsmath}
\usepackage{amssymb}

\begin{document}

\begin{align*}
\frac{4}{\pi^2} = \sum_{n=0}^{\infty} (-1)^n r(n) \frac{5(13 + 180n + 820n^2)}{32 \cdot 2^{n+1}}
\frac{2}{\pi^2} = \sum_{n=0}^{\infty} (-1)^n r(n) \frac{5(1 + 8n + 20n^2)}{2 \cdot 2^{n+1}}
 r(n) := \frac{\displaystyle\prod_{i=1}^{n} \left(\frac{2i-1}{2i}\right)}{n!} \newline
\text{Donde }r(n) \text{ se define como:}
\end{align*}

\end{document}
