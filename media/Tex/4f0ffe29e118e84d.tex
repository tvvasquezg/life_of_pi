\documentclass[preview]{standalone}

\usepackage[english]{babel}
\usepackage{amsmath}
\usepackage{amssymb}

\begin{document}

\begin{center}
Luego hizo el mar de fundición, que tenía diez codos de borde a borde; era completamente redondo, su altura era de cinco codos y una línea de treinta codos lo rodeaba.Acá se deduce que se tomaba $\pi$ como $\frac{30}{10}$
\end{center}

\end{document}
