\documentclass[preview]{standalone}

\usepackage[english]{babel}
\usepackage{amsmath}
\usepackage{amssymb}

\begin{document}

\begin{center}
Históricamente, las aproximaciones de $\pi$ han variado, con los babilonios utilizando 3.125 y los egipcios $\frac{256}{81}$ que es igual a 3.1604.\n
También se ha propuesto que los hebreos usaron $\pi$ igual a 3 basándose en un verso bíblico I de Reyes 7:23 que dice: ``Luego hizo el mar de fundición, que tenía diez codos de borde a borde; era completamente redondo, su altura era de cinco codos y una línea de treinta codos lo rodeaba.'' Acá se deduce que se tomaba $\pi$ como $\frac{30}{10}$.
\end{center}

\end{document}
